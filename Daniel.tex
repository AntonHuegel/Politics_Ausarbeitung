% Teil Daniel

\definecolor{rahmen}{rgb}{1,.7,.7}

\section{Akquirierung von Texten}
Um politische Texte analysieren zu können müssen zuallererst passende Texte beschafft werden. Quellen für solche Texte sind verschiedene vorhanden, welche mehr oder weniger gut geeignet sind, um automatisch analysiert zu werden. Im Folgenden werden drei Kategorien kurz vorgestellt.

\subsection{Textbasiert}
Rein auf den gedruckten Text ausgelegte Medien, wie beispielsweise Zeitungen oder Magazine, sind für den Computer im Allgemeinen schwer auszuwerten. Die Daten müssten manuell eingescannt werden und anschließend per Bildbearbeitungsprogramm in ein geeignetes Zeichenformat übersetzt werden. 

\subsection{Audio/Video}
Medien, welche Nachrichten per Audio oder Video übertragen, sind ebenfalls relativ uninteressant für die automatische Analyse. Hierzu zählen beispielsweise Fernsehsendungen, Radiosender oder auch Podcasts im Internet. Ähnlich wie bei den rein gedruckten Medien besteht auch hier das Problem, die Textinformationen in ein für den Computer „leserliches“ Format zu konvertieren. Insbesondere bei akustischen Signalen, also gesprochenem Text, ist diese Analyse sehr aufwändig und Fehleranfällig. 

\subsection{Internet}	

\section{Datenbanken}
Im Zuge der Textverarbeitung trifft man immer wieder auf den Begriff der Datenbanken. Diese sind im ein wichtiges Werkzeug, um die Textmanipulation sinnvoll zu unterstützen. Eine Datenbank stellt die nötigen Funktionen zur Verfügung, um große Datenmengen effizient zu speichern, zu analysieren und zu verarbeiten. Im Bereich der relationalen Datenbanken ist die Sprache SQL (Structured Query Language) weitgehend verbreitet. Diese bietet unterschiedliche Möglichkeiten an, die gespeicherten Daten abzufragen und zu manipulieren:
\begin{enumerate}
\item INSERT
\item UPDATE
\item SELECT
\item SORT
\item JOIN
\item DELETE
\end{enumerate}
Logische Bedingungen können, wie in anderen Sprachen auch, mit in die Abfragen integriert werden. Diese werden durch einige zusätzliche Funktionen unterstützt. Beispielsweise lassen sich die Länge von Texten oder Mittelwerte schnell berechnen und in die logische Bedingung integrieren.
Wichtig hierbei ist ebenfalls der Begriff der Normalisierung. Um Konsitenz zu wahren und Redundanzen zu vermeiden existieren verschiedene Normalformen:
\begin{enumerate}
\item 1NF - Erste Normalform
\item 2NF - Zweite Normalform
\item 3NF - Dritte Normalform
\item BCNF - Boyce-Codd Normalform
\end{enumerate}
Die Formen sind hierbei anhand ihrer Striktheit aufsteigend angeordnet. Als ein Beispiel aus dem Kontext der politischen Textanalyse wird ein kurzes Schema zur Speicherung politischer Texte vorgestellt. Relevante Informationen sind hierbei die Herkunft des Textes (Politiker, Partei) sowie eine eindeutige Zuordnung anhand einer Identifikationsnummer. \\
\begin{tabular}{llllll}
\hline
ID & Partei & Kürzel & Vorname & Nachname & Text \\
\hline
1 & Christlich Demokratische Union & CDU & Angela & Merkel & Text1 \\
2 & Christlich Demokratische Union & CDU & Angela & Merkel & Text2 \\
3 & Christlich Demokratische Union & CDU & Angela & Merkel & Text3 \\
4 & Christlich Demokratische Union & CDU & Angela & Merkel & Text4 \\
5 & Sozialdemokratische Partei Deutschlands & SPD & Sigmar & Gabriel & Text5 \\
\hline
\end{tabular}
Offensichtlich werden hier die Daten der Partei sowie der Politiker redundant gespeichert, was suboptimal ist. Um einer möglichst hohen Normalform zu entsprechen wird die Tabelle in drei Tabellen aufgespalten: Je eine Tabelle für die Parteien, die Politiker und die Texte: \\
\begin{tabular}{lll}
\hline
ID & Partei & Kürzel \\
\hline
1 & Christlich Demokratische Union & CDU \\
2 & Sozialdemokratische Partei Deutschlands & SPD \\
\hline
\end{tabular}
\\
\begin{tabular}{llll}
\hline
ID & Vorname & Nachname & Partei-ID\\
\hline
1 & Angela & Merkel & 1\\
2 & Sigmar & Gabriel & 2\\
\hline
\end{tabular}
\begin{tabular}{lll}
\hline
ID & Politiker-ID & Text \\
\hline
1 & 1 & Text1 \\
2 & 1 & Text2 \\
3 & 1 & Text3 \\
4 & 1 & Text4 \\
5 & 2 & Text5 \\
\hline
\end{tabular}
\\
Die Identifikationsnummer aller drei Tabellen kommt hierbei auch als Schlüssel ins Spiel. Über den Schlüssel bzw. Index können Einträge in einer Tabelle eindeutig identifiziert werden, was sehr effiziente Abfragen ermöglicht. Andererseits werden eben jene IDs in anderen Tabellen als sogenannte Fremdschlüssel verwendet: Sie zeigen auf Schlüssel anderer Tabellen, um Redundanzen zu vermeiden. In obigem Beispiel sind die \textit{Partei-ID} und die \textit{Politiker-ID} Fremdschlüssel. Im Hinblick auf Fremdschlüssel ist besonders auf die Konsistenz der Datensätze zu achten. Wäre beispielsweise der Politiker \textit{Sigmar Gabriel} nicht vorhanden, also die \textit{Politiker-ID} 2 nicht vergeben, so würden die anderen Tabellen auf ungültige Datensätze verweisen.

\section{Textvorverarbeitung}

Da die Texte, so wie sie im Internet zur Verfügung gestellt werden, für das Lesen durch den Menschen optimiert sind, müssen noch einige zusätzliche Schritte ausgeführt werden, damit die Daten auch effektiv durch den Computer ausgewertet werden können. Anhand von zwei kurzen Beispielen wird die Notwendigkeit hierfür deutlich:

\colorbox{rahmen}{"Wir} wollen scheinbare \colorbox{rahmen}{Gegensätze} verbinden\colorbox{rahmen}{. Umweltschutz} muss nicht wie bei \colorbox{rahmen}{Rot-Grün} die \colorbox{rahmen}{Wirtschaft} fesseln\colorbox{rahmen}{. Man} kann die \colorbox{rahmen}{Sicherheit stärken}, ohne \colorbox{rahmen}{Bürgerrechte einzuschränken. Soziale Verantwortung übernimmt} man besser mit \colorbox{rahmen}{Bildung} als mit \colorbox{rahmen}{Umverteilung.}"

\colorbox{rahmen}{wir} wollen scheinbare \colorbox{rahmen}{gegensaetze} verbinden umweltschutz muss nicht wie bei rot gruen \colorbox{rahmen}{die} wirtschaft fesseln man \colorbox{rahmen}{kann die} sicherheit staerken ohne buergerrechte einzuschraenken soziale verantwortung \colorbox{rahmen}{uebernimmt} man \colorbox{rahmen}{besser} mit bildung als mit umverteilung

\colorbox{rahmen}{ich} wollen scheinbare gegensatz verbinden umweltschutz muss nicht \colorbox{rahmen}{wie} bei rot gruen \colorbox{rahmen}{der} wirtschaft fesseln \colorbox{rahmen}{man} koennen \colorbox{rahmen}{der} sicherheit staerken \colorbox{rahmen}{ohne} buergerrechte einschraenken soziale verantwortung uebernehmen \colorbox{rahmen}{man} gut \colorbox{rahmen}{mit} bildung \colorbox{rahmen}{als mit} umverteilung

wollen scheinbare gegensatz verbinden umweltschutz muss nicht rot gruen wirtschaft fesseln koennen sicherheit staerken buergerrechte einschraenken soziale verantwortung uebernehmen gut bildung umverteilung

\subsection{Zeichenkodierung}
\subsection{Reguläre Ausdrücke	}
Reguläre Ausdrücke werden in der Informationsverarbeitung bestimmte Muster aus Texten erkennen und extrahieren zu können. Das zu erkennende Muster wird hierbei ebenfalls als Zeichenkette mit spezieller Syntax übergeben. Hierbei werden über lateinische Zeichen, arabische Ziffern bis hin zu diversen Sonderzeichen die meisten Anwendungsfälle abgedeckt.
Eine sehr häufige Verwendung finden reguläre Ausdrücke im Bereich der Validierung von Nutzereingaben auf Webseiten, insbesondere in Formularen zur Datenübergabe. Einige bekannte Beispiele hierfür sind:
\newline
\begin{tabular}{llll}
\hline
Name & Ausdruck & Beispiel \\
\hline
Postleitzahl & /[0-9]{5}/ & 87700 \\
Vorname/Nachname & /[a-zA-Z]+/ & Peter Lustig \\
E-Mail & /[a-zA-Z]+@[a-zA-Z]+.(de|com|net)/ & anonymous@uni-ulm.de \\
Telefonnummer & /[0-9]+[ -]{1,3}[0-9 ]{4,}/ & 0190 - 123 456 \\
\hline
\end{tabular}

\subsection{Deutsche Grammatik	}
Im deutschen lassen sich Worte durch sogenannte Wortarten kategorisieren. Zu den Wortarten zählen: \\
\begin{tabular}{ll}
\hline
Wort & Beispiel \\
\hline
Nomen & Politiker, Partei \\
Artikel & der, die \\
Pronomen & dieser, dieses \\
Adjektive & rot, liberal \\
Verben & wählen, senken \\
Adverbien & schnell, laut \\
Konjunktionen & und, oder \\
Präpositionen & an, auf \\
\hline
\end{tabular} \\
Im allgemeinen können viele Sätze auf eine Grundform reduziert werden: Subjekt - Verb - Objekt. Anhand dieser Form ist es für ein Programm einfach,
einen Aussage aus einem Satz herauszuarbeiten. Da Sätze jedoch häufig verschachtelt und detaillierter ausformuliert sind, müssen diese weitestgehend vereinfacht
werden. Ein Schritt hierzu ist das Entfernen nicht-sinngebender Wortarten aus dem Text, also solcher Worte, die selbst keine essentielle Information beitragen 
sondern lediglich zur Wahrung der grammatikalischen Korrektheit im Text vorhanden sind. Die Wortarten im Text reduzieren sich damit auf Nomen, Adjektive, Verben
und Adverbien.
Ein zweiter Schritt ist das Reduzieren eben jener Wortformen auf ihre grammatikalische Grundform. Die Änderung ab von der Grundform nennt man bei Verben Konjugation
und bei Nomen Deklination. Die Variation der Wörter hängt hierbei von vielen verschiedenen Faktoren ab, beispielsweise der Zeitform oder der Person.

\subsection{Morphologie Datenbank	}
\subsection{Synonym Datenbank}
\section{Comparative Manifesto Project}
Das Comparative Manifesto Project, kurz CMP, beschäftigt sich mit der Analyse politischer Texte und Standpunkte einer Vielzahl von Ländern und Parteien weltweit. Als Resultat stehen dem Nutzer der CMP Datenbank ausgiebige Informationen über die jeweiligen Texte zur Verfügung - und zwar in maschinell verwertbarer Form.
Das Auswerten der Texte erfolgt jedoch nicht durch ein Programm, sondern manuell durch eine effiziente Aufteilung auf verschiedene Arbeiter. Dieses Vorgehen wird Crowd-Sourcing genannt. Dabei werden die Texte in kompakte Pakete aufgeteilt und weltweit an verschiedene Arbeiter verteilt. Dieses Vorgehen hat einige Vorteile:
\begin{enumerate}
\item Der Arbeiter hat nicht den ganzen Text zur Verfügung und muss daher den Teiltext objektiv und ohne Zusammenhang einordnen. Im Kontext des gesamten Textes könnten sonst subjektive Eindrücke das Ergebnis verfälsche.
\item Identische Teiltexte können an unterschiedliche Arbeiter verteilt werden. So lassen sich subjektive Einstellungen der Arbeiter herausfiltern. Für eine steigende Zahl an Arbeitern kann so ein möglichst objektives Ergebnis erzielt werden.
\end{enumerate}
Als Ergebnis der Teilarbeit erhält man nun die Einordnung des Textes in ein Schema aus 56 Kategorien und Subkategorien. Diese Kategorien sollen das ganze politische Spektrum sinnvoll abbilden können. Die Hauptkategorien hierbei sind:
\begin{enumerate}
\item External Relations
\item Freedom and Democracy
\item Political System
\item Economy
\item Welfare and Quality of Life
\item Fabric of Society
\item Social Groups
\end{enumerate}
Diese lassen sich wiederum in Subkategorien einteilen um das Ergebnis detaillierter darstellen zu können. Da jede Aussage eines Textes solch einer Kategorie zugeordnet wird, erhält man so eine 56-dimensionale Darstellung des Textes, wobei jede Dimension eine politische Richtung darstellt.
Das Manifesto Project enthält seit 1945 ungefähr 4000 Texte von über 1000 Parteien aus mehr als 50 verschiedenen Ländern.
Die Daten können manuell über eine minimalistische Oberfläche abgefragt werden. Alternativ steht, nach einer initialen Registrierung, auch eine API zur automatischen Abfrage zur Verfügung.

\section{Repräsentative Demokratie in Deutschland}
\begin{quote}
"Die repräsentative Demokratie bezeichnet eine demokratische Herrschaftsform, bei der politische Entscheidungen und die Kontrolle der Regierung nicht unmittelbar vom Volk, sondern von einer Volksvertretung, zum Beispiel dem Parlament, ausgeübt werden.
Bürgerinnen und Bürger treffen politische Entscheidungen nicht selbst, sondern überlassen sie auf Zeit gewählten Vertretern, die für sie als Stellvertreter tätig sind. Die Bürger beteiligen sich aber an Wahlen und wirken in Parteien, Verbänden und Initiativen mit."
\end{quote}
\cite{bundesregierung}
Die Bundesregierung in Deutschland setzt sich aus aus der Bundeskanzlerin und den verschiedenen Bundesministerien zusammen. Aktuell besteht die Regierung aus einer Koalition von CDU/CSU und SPD. Eine kurze Übersicht bietet folgende Tabelle: \\
\begin{tabular}{lll}
\hline
Name & Partei & Rolle \\
\hline
Angela Merkel & CDU & Bundeskanzlerin der Bundesrepublik Deutschland \\
Sigmar Gabriel & SPD & Bundesminister des Auswärtigen \\
Brigitte Zypries & SPD & Bundesministerin für Wirtschaft und Energie \\
Thomas de Maizière & CDU & Bundesminister des Innern \\
Heiko Maas & SPD & Bundesminister der Justiz und für Verbraucherschutz \\
Wolfgang Schäuble & CDU & Bundesminister der Finanzen \\
Andrea Nahles & SPD & Bundesministerin für Arbeit und Soziales \\
Christian Schmidt & CSU & Bundesminister für Ernährung und Landwirtschaft \\
Ursula von der Leyen & CDU & Bundesministerin der Verteidigung \\
Katarina Barley  & SPD & Bundesministerin für Familie, Senioren, Frauen und Jugend \\
Hermann Gröhe & CDU & Bundesminister für Gesundheit \\
Alexander Dobrindt & CSU & Bundesminister für Verkehr und digitale Infrastruktur \\
Barbara Hendricks & SPD & Bundesministerin für Umwelt, Naturschutz, Bau und Reaktorsicherheit \\
Johanna Wanka & CDU & Bundesministerin für Bildung und Forschung \\
Gerd Müller & CSU & Bundesminister für wirtschaftliche Zusammenarbeit und Entwicklung \\
Peter Altmaier & CDU & Chef des Bundeskanzleramtes und Bundesminister für besondere Aufgaben \\
\hline
\end{tabular} \\
Im Gegensatz dazu sind im Bundestag alle gewählten Parteien vertreten, die über die "Fünf Prozent Hürde" gekommen sind. Dazu gehören, abgesehen von CDU/CSU und SPD, Die Linke und BÜNDNIS 90/DIE GRÜNEN. Weitere wichtige Parteien im politischen Spektrum sind die FDP und die AfD. Wichtige Personen in diesem Zusammenhang sind die Parteivorsitzenden und deren Stellvertreter: \\
\begin{tabular}{lll}
\hline
Partei & Parteivorsitzender & stellvertretende Vorsitzende \\
\hline
CDU & Angela Merkel & Volker Bouffier, Ursula von der Leyen, Julia Klöckner, Armin Laschet, Thomas Strobl \\
CSU & Horst Seehofer & Manfred Weber, Christian Schmidt, Barbara Stamm, Angelika Niebler, Kurt Gribl \\
SPD & Martin Schulz & Thorsten Schäfer-Gümbel, Manuela Schwesig, Olaf Scholz, Aydan Özoğuz, Ralf Stegner \\
Bündnis 90/Die Grünen & Cem Özdemir, Simone Peter & \\
Die Linke & Katja Kipping, Bernd Riexinger & Caren Lay, Axel Troost, Tobias Pflüger, Janine Wissler \\
FDP & Christian Lindner & Wolfgang Kubicki, Marie-Agnes Strack-Zimmermann, Katja Suding \\
AfD & Frauke Petry, Jörg Meuthen & Alexander Gauland, Beatrix von Storch, Albrecht Glaser \\
\hline
\end{tabular} \\
Abgesehen von den oben genannten Politikern gibt es noch weitere Ämter, wie beispielsweise Fraktionsvorsitzende oder Generalsekretäre, welche ebenfalls eine wichtige Informationsquelle darstellen, hier aber nicht explizit aufgeführt werden. Bis auf wenige Ausnahmen sind alle oben genannten Politiker in den sozialen Medien aktiv und es ist somit möglich, poltische Texte abzufragen. Da diese durch die professionellen Büros der Politiker erstellt werden, sind sie in angebrachter Sprache und Grammatik verfasst. Somit sind sie auch für die Analyse mit Hilfe des Modells verwendbar, da sich der Wortschatz ähneln sollte.
