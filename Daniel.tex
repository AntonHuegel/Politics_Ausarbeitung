% Teil Daniel

\section{Akquirierung von Texten}
Um politische Texte analysieren zu können müssen zuallererst passende Texte beschafft werden. Quellen für solche Texte sind verschiedene vorhanden, welche mehr oder weniger gut geeignet sind, um automatisch analysiert zu werden. Im Folgenden werden drei Kategorien kurz vorgestellt.

\subsection{Textbasiert}
Rein auf den gedruckten Text ausgelegte Medien, wie beispielsweise Zeitungen oder Magazine, sind für den Computer im Allgemeinen schwer auszuwerten. Die Daten müssten manuell eingescannt werden und anschließend per Bildbearbeitungsprogramm in ein geeignetes Zeichenformat übersetzt werden. 

\subsection{Audio/Video}
Medien, welche Nachrichten per Audio oder Video übertragen, sind ebenfalls relativ uninteressant für die automatische Analyse. Hierzu zählen beispielsweise Fernsehsendungen, Radiosender oder auch Podcasts im Internet. Ähnlich wie bei den rein gedruckten Medien besteht auch hier das Problem, die Textinformationen in ein für den Computer „leserliches“ Format zu konvertieren. Insbesondere bei akustischen Signalen, also gesprochenem Text, ist diese Analyse sehr aufwändig und Fehleranfällig. 

\subsection{Internet}	

\section{Datenbanken}
Im Zuge der Textverarbeitung trifft man immer wieder auf den Begriff der Datenbanken. Diese sind im ein wichtiges Werkzeug, um die Textmanipulation sinnvoll zu unterstützen. Eine Datenbank stellt die nötigen Funktionen zur Verfügung, um große Datenmengen effizient zu speichern, zu analysieren und zu verarbeiten. Im Bereich der relationalen Datenbanken ist die Sprache SQL (Structured Query Language) weitgehend verbreitet. Diese bietet unterschiedliche Möglichkeiten an, die gespeicherten Daten abzufragen und zu manipulieren:
\begin{enumerate}
\item INSERT
\item UPDATE
\item SELECT
\item SORT
\item JOIN
\item DELETE
\end{enumerate}
Logische Bedingungen können, wie in anderen Sprachen auch, mit in die Abfragen integriert werden. Diese werden durch einige zusätzliche Funktionen unterstützt. Beispielsweise lassen sich die Länge von Texten oder Mittelwerte schnell berechnen und in die logische Bedingung integrieren.

\section{Textvorverarbeitung}
    \subsection{Zeichenkodierung}
    \subsection{Reguläre Ausdrücke	}
    \subsection{Deutsche Grammatik	}
    \subsection{Morphologie Datenbank	}
    \subsection{Synonym Datenbank}

\section{CMP Manifesto Project}